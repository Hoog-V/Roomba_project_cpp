{\bfseries{F-\/\+REQ 1 \mbox{[}MH\mbox{]}\+:}} De \mbox{\hyperlink{namespace_roomba}{Roomba}} library broncode moet het mogelijk maken om de richting van de \mbox{\hyperlink{namespace_roomba}{Roomba}} te kunnen wijzigen in rechtdoor, achteruit of een bepaalde draairichting met meegegeven hoek.

{\bfseries{F-\/\+REQ 2 \mbox{[}MH\mbox{]}\+:}} De \mbox{\hyperlink{namespace_roomba}{Roomba}} library moet de ingebouwde sensoren van de \mbox{\hyperlink{namespace_roomba}{Roomba}} kunnen uitlezen.

{\bfseries{F-\/\+REQ 3 \mbox{[}MH\mbox{]}\+:}} De \mbox{\hyperlink{namespace_roomba}{Roomba}} library broncode moet het mogelijk maken om de control modus van de \mbox{\hyperlink{namespace_roomba}{Roomba}} aan te passen. De modi die het moet ondersteunen zijn\+:
\begin{DoxyItemize}
\item Passive
\item Safe
\item Full
\end{DoxyItemize}

{\bfseries{F-\/\+REQ 4 \mbox{[}SH\mbox{]}\+:}} De \mbox{\hyperlink{namespace_roomba}{Roomba}} library broncode moet het mogelijk maken om een schoonmaak ronde te starten op de \mbox{\hyperlink{namespace_roomba}{Roomba}}. De schoonmaak modi die het moet ondersteunen zijn\+:
\begin{DoxyItemize}
\item Normale schoonmaak
\item volledige schoonmaak
\item schoonmaak op de huidige plek (spot cleaning) zijn.
\end{DoxyItemize}

{\bfseries{F-\/\+REQ 5 \mbox{[}SH\mbox{]}\+:}} De \mbox{\hyperlink{namespace_roomba}{Roomba}} library broncode moet het mogelijk maken om de dock modus van de \mbox{\hyperlink{namespace_roomba}{Roomba}} te activeren.

{\bfseries{F-\/\+REQ 6 \mbox{[}SH\mbox{]}\+:}} De \mbox{\hyperlink{namespace_roomba}{Roomba}} library broncode moet het mogelijk maken om de \mbox{\hyperlink{namespace_roomba}{Roomba}} aan en uit te zetten.

{\bfseries{F-\/\+REQ 7 \mbox{[}SH\mbox{]}\+:}} De \mbox{\hyperlink{namespace_roomba}{Roomba}} library broncode moet het mogelijk maken om de motoren van de \mbox{\hyperlink{namespace_roomba}{Roomba}} te manipuleren. De motoren die het moet kunnen manipuleren zijn\+:
\begin{DoxyItemize}
\item Borstel
\item Zuiger
\item Zijborstel
\end{DoxyItemize}

{\bfseries{F-\/\+REQ 8 \mbox{[}SH\mbox{]}\+:}} De \mbox{\hyperlink{namespace_roomba}{Roomba}} library broncode moet het mogelijk maken om de leds van de \mbox{\hyperlink{namespace_roomba}{Roomba}} te manipuleren. De leds die het moet kunnen manipuleren zijn\+:
\begin{DoxyItemize}
\item DD
\item MAX
\item CLEAN
\item SPOT
\item STATUS
\end{DoxyItemize}

{\bfseries{F-\/\+REQ 9 \mbox{[}SH\mbox{]}\+:}} De \mbox{\hyperlink{namespace_roomba}{Roomba}} library broncode moet het mogelijk maken om de Baudrate van de SCI poort te wijzigen. De baudrates die het moet ondersteunen zijn\+:
\begin{DoxyItemize}
\item 300 bps
\item 600 bps
\item 1200 bps
\item 2400 bps
\item 4800 bps
\item 9600 bps
\item 14400 bps
\item 19200 bps
\item 28800 bps
\item 38400 bps
\item 57600 bps
\item 115200 bps
\end{DoxyItemize}

{\bfseries{F-\/\+REQ 10 \mbox{[}SH\mbox{]}\+:}} De \mbox{\hyperlink{namespace_roomba}{Roomba}} library broncode moet het mogelijk maken om noten af te spelen op de \mbox{\hyperlink{namespace_roomba}{Roomba}}.

{\bfseries{F-\/\+REQ 11 \mbox{[}CH\mbox{]}\+:}} De \mbox{\hyperlink{namespace_roomba}{Roomba}} library broncode moet liedjes af kunnen spelen op de \mbox{\hyperlink{namespace_roomba}{Roomba}} met behulp van midi bestanden.

Moscow methode\+: \mbox{[}MH\mbox{]} = Must Have \mbox{[}SH\mbox{]} = Should Have \mbox{[}CH\mbox{]} = Could Have

{\bfseries{NF-\/\+REQ 1 \mbox{[}MH\mbox{]}\+:}} De \mbox{\hyperlink{namespace_roomba}{Roomba}} library moet zo platform onafhankelijk mogelijk geschreven worden.

{\bfseries{NF-\/\+REQ 2 \mbox{[}MH\mbox{]}\+:}} De \mbox{\hyperlink{namespace_roomba}{Roomba}} library moet in de programmeertaal C++ geschreven worden. 